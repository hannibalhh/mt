\documentclass[10pt,a4paper]{article}
\usepackage[utf8]{inputenc}
\usepackage[german]{babel}
\usepackage[T1]{fontenc}
\usepackage{amsmath}
\usepackage{amsfonts}
\usepackage{amssymb}
\usepackage{float}
\usepackage{graphicx}
\renewcommand\thesubsection{\alph{subsection}}
\begin{document}

\begin{center}
Modellierung hybrider Systeme mit Stateflow: Praktikum 3

Johannes Reidl, Sigurd Sippel

\end{center}

\section{Start einer zweistufigen Rakete}

	\begin{figure}[H]
 	 		\centering
 	 		\includegraphics[width=1\textwidth]{../aufgabe1/screens/rocket_outer.png}
 	 		\caption{Simulink Stateflow}
 	 	\end{figure}

	\begin{figure}[H]
 	 		\centering
 	 		\includegraphics[width=1\textwidth]{../aufgabe1/screens/rocket_para.png}
 	 		\caption{Simulink Stateflow}
 	 	\end{figure}

		\begin{figure}[H]
 	 		\centering
 	 		\includegraphics[width=1\textwidth]{../aufgabe1/screens/rocket_state.png}
 	 		\caption{Simulink States}
 	 	\end{figure}
 	 	
 	 	\begin{align}
 	 	Acc1 = 
 	 	\\mR = m1 + m2 \nonumber
 	 	\\r1 = x1 + rE \nonumber
 	 	\\Fs1 = G * (mE * mR) / (r1*r1) \nonumber
 	 	\\Schubkraft_1 = Durchsatz_1 * SchubProDurchsatz \nonumber
 	 	\\a = (Schubkraft_1 - Fs1) / (m1+m2) \nonumber
 	 	\\a1 = a \nonumber
 	 	\\a2 = a \nonumber
 	 	\end{align}
 	 	
 	 	\begin{align}
 	 	Acc2 =	 	 
 	 	\\r1 = x1 + rE \nonumber
 	 	\\Fs1 = G * (mE * m1) / (r1*r1) \nonumber
 	 	\\r2 = x2 + rE \nonumber
 	 	\\Fs2 = G * (mE * m2) / (r2*r2) \nonumber
 	 	\\Schubkraft_2 = Durchsatz_2 * SchubProDurchsatz \nonumber
 	 	\\a1 = - Fs1 / m1 \nonumber
 	 	\\a2 = (Schubkraft_2- Fs2) / m2 \nonumber
 	 	\end{align}

\begin{align}
Acc3 =
\\r1 = x1 + rE \nonumber
\\Fs1 = G * (mE * m1) / (r1*r1) \nonumber
\\r2 = x2 + rE \nonumber
\\Fs2 = G * (mE * m2) / (r2*r2) \nonumber
\\a1 = - Fs1 / m1 \nonumber
\\a2 = - Fs2 / m2 \nonumber
\end{align}

		 		\begin{figure}[H]
	 	 	 	 		\centering
	 	 	 	 		\includegraphics[width=1\textwidth]{../aufgabe1/screens/geschwindigkeiten.png}
	 	 	 	 		\caption{Geschwindigkeiten}
	 	 	 	 	\end{figure}
		 		\begin{figure}[H]
		 	 	 	 		\centering
		 	 	 	 		\includegraphics[width=1\textwidth]{../aufgabe1/screens/hoehen.png}
		 	 	 	 		\caption{Höhen}
		 	 	 	 	\end{figure}
\section{Simulation eines schiefen Flippers}

\end{document}
