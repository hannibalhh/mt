\documentclass[10pt,a4paper]{article}
\usepackage[utf8]{inputenc}
\usepackage[german]{babel}
\usepackage[T1]{fontenc}
\usepackage{amsmath}
\usepackage{amsfonts}
\usepackage{amssymb}
\usepackage{float}
\usepackage{graphicx}
\usepackage{listings}
\renewcommand\thesubsection{\alph{subsection}}
\begin{document}

\begin{center}
Regelung dynamischer Systeme: Praktikum 4

Johannes Reidl, Sigurd Sippel
\end{center}

\section{Schwebekugel}
\subsection{Bestimmung von $i_0$ und  $u_o$ im Arbeitspunkt}
\begin{figure}[H]
  \begin{align}
	  m\ddot{x} = mg - C * (\frac{i}{x})^2 \nonumber \\
	  \frac{m\ddot{x} - mg}{-c} = (\frac{i}{x})^2 \nonumber \\
	  \sqrt{\frac{m* (\ddot{x} - g)}{-c} * x} = i \nonumber \\
	  \sqrt{\frac{0,025kg * -9,81\frac{m}{s^2}}{-5 * 10^{-6}\frac{Nm^2}{A^2}}} * 0,015m = i \nonumber \\
	  \underline{\underline{i = 3,322A}} \nonumber
  \end{align}
\end{figure}

\begin{figure}[H]
  \begin{align}
	  u(t) = R * i(t) + L * \underbrace{\frac{di(t)}{dt}}_\text{= 0, da Kugel in Ruhe} \nonumber \\
	  u(t) = 3\frac{V}{A} * 3,322A = \underline{\underline{9,966V}} \nonumber
  \end{align}
\end{figure}
\subsection{Strukturbilder der DGLn}


\subsection{Linearisierung}


\subsection{Normieren der linearisierten DGLn auf SI-Größen}


\subsection{Übertragungsfunktionen}


\subsection{Gesamtübertragungsfunktion der Regelstrecke}


\end{document}
